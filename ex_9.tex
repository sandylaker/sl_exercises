%% LaTeX Beamer presentation template (requires beamer package)
%% see http://bitbucket.org/rivanvx/beamer/wiki/Home
%% idea contributed by H. Turgut Uyar
%% template based on a template by Till Tantau
%% this template is still evolving - it might differ in future releases!

%% Template edited by Panagiotis Adamopoulos {padamopo}@stern.nyu.edu

\documentclass[aspectratio=169]{beamer}
 
\mode<presentation>
{
\usetheme{NYU}

\setbeamercovered{transparent}
}

\usepackage[english]{babel}
\usepackage[latin1]{inputenc}
\usepackage[scaled=.90]{helvet}
\usepackage{courier}
\usepackage[T1]{fontenc}
\usepackage{comment}
%usepackage{appendixnumberbeamer}
\usepackage{amsmath}
\usepackage{bm}
\usepackage{pgfpages}
% citations
\usepackage{natbib}
\bibpunct{(}{)}{;}{a}{,}{,}
\def\citeapos#1{\citeauthor{#1}'s (\citeyear{#1})}
\renewcommand{\bibsection}{\subsubsection*{\bibname } }

\input{latex-math/basic-math.tex}
\input{latex-math/basic-ml.tex}
\input{latex-math/ml-svm.tex}

\newcommand{\zv}{\bm{z}}


\title[]{\textbf{Exercise of Supervised Learning: \\ SVM Part 2}}

%\subtitle{}

% - Use the \inst{?} command only if the authors have different
%   affiliation.
%\author{F.~Author\inst{1} \and S.~Another\inst{2}}
\author{Yawei Li} 

% - Use the \inst command only if there are several affiliations.
% - Keep it simple, no one is interested in your street address.
\institute[LMU]
{
\\
  \texttt{yawei.li@stat.uni-muenchen.de}
}

\date{December 15, 2023}


% This is only inserted into the PDF information catalog. Can be left
% out.
\subject{Subject}



% If you have a file called "university-logo-filename.xxx", where xxx
% is a graphic format that can be processed by latex or pdflatex,
% resp., then you can add a logo as follows:

% \pgfdeclareimage[height=0.5cm]{university-logo}{university-logo-filename}
% \logo{\pgfuseimage{university-logo}}



% Delete this, if you do not want the table of contents to pop up at
% the beginning of each subsection:
%\AtBeginSubsection[]
%{
%\begin{frame}<beamer>
%\frametitle{Outline}
%\tableofcontents[currentsection,currentsubsection]
%\end{frame}
%}

% If you wish to uncover everything in a step-wise fashion, uncomment
% the following command:

%\beamerdefaultoverlayspecification{<+->}

\begin{document}



\begin{frame}[noframenumbering, plain]
\titlepage

\end{frame}

\begin{frame}{Exercise 1: Kernelized Multiclass SVM}
	\small
	For a data set $\D = \{(\xv^{(1)}, y^{(1)}), \ldots, (\xv^{(n)}, y^{(n)}) \}$ with $\yi \in \Yspace = \{+1, -1 \}$, assume that we are provided with a suitable feature map $\phi: \Xspace \to \Phi$, where $\Phi \subset \mathbb{R}^d$. In the featureized SVM learning problem we are facing the following optimization problem:
	\begin{align*}
		\min_{\thetab, \theta_0, \sli} &\frac{1}{2} \thetab^T \thetab + C \sumin \sli \\
		\text{s.t. } & \yi \left(\left\langle \thetab, \phixi + \theta_0 \right\rangle \right) \geq 1 - \sli \qquad \forall i \in \{1, \ldots, n \}, \\
		\text{and } & \sli \geq 0 \qquad i \in \{1, \ldots, n \},
	\end{align*}
	where $C \geq 0$ is some constant.
	
	(a) Argue that this is equivalent to the following ERM problem:
		\begin{align*}
			\riske(\thetab) = \frac{1}{2} ||\thetab||^2 + C\sumin \max(1 - \yi(\thetab^T \phixi + \theta_0)), 0).
		\end{align*}
		i.e., the regularized ERM problem for the hinge loss for the hypothesis space 
		\begin{align*}
			\Hspace = \{f: \Phi \to \mathbb{R} \ | \ f(\zv) = \thetab^T \zv + \theta_0, \thetab \in \mathbb{R}^d, \theta_0 \in \mathbb{R} \}
		\end{align*}
\end{frame}

\begin{frame}{Solution to Exercise 1 (a)}
	
\end{frame}

\end{document}
