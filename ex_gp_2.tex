%% LaTeX Beamer presentation template (requires beamer package)
%% see http://bitbucket.org/rivanvx/beamer/wiki/Home
%% idea contributed by H. Turgut Uyar
%% template based on a template by Till Tantau
%% this template is still evolving - it might differ in future releases!

%% Template edited by Panagiotis Adamopoulos {padamopo}@stern.nyu.edu

\documentclass[aspectratio=169]{beamer}
 
\mode<presentation>
{
\usetheme{NYU}

\setbeamercovered{transparent}
}

\usepackage[english]{babel}
\usepackage[latin1]{inputenc}
\usepackage[scaled=.90]{helvet}
\usepackage{courier}
\usepackage[T1]{fontenc}
\usepackage{comment}
%usepackage{appendixnumberbeamer}
\usepackage{amsmath}
\usepackage{bm}
\usepackage{pgfpages}
% citations
\usepackage{natbib}
\bibpunct{(}{)}{;}{a}{,}{,}
\def\citeapos#1{\citeauthor{#1}'s (\citeyear{#1})}
\renewcommand{\bibsection}{\subsubsection*{\bibname } }

\input{latex-math/basic-math.tex}
\input{latex-math/basic-ml.tex}
\input{latex-math/ml-gp.tex}

\renewcommand{\N}{\mathcal{N}}
\renewcommand{\R}{\mathbb{R}}
\newcommand{\kij}{\bm{K}_{ij}}
\renewcommand{\kxij}{k(x^{(i)}, x^{(j)})}
\newcommand{\fvs}{\fv_{*}}
\newcommand{\xvs}{\xv_{*}}
\newcommand{\ssqiinv}{(\sigma^2 \id)^{-1}}
\newcommand{\kinv}{\Kmat^{-1}}
\newcommand{\kpinv}{\Kmat_{\text{post}}^{-1}}
\newcommand{\tfv}{\tilde{\bm{f}}}
\newcommand{\bv}{\bm{b}}
\newcommand{\Kpost}{\bm{K}_{\text{post}}}
\newcommand{\fpost}{\fv_{\text{post}}}
\newcommand{\ys}{y_{*}}
\newcommand{\xs}{x_{*}}
\newcommand{\ms}{m_{*}}
\newcommand{\av}{\bm{a}}

\title[]{\textbf{Exercise of Supervised Learning: \\ Gaussian Processes Part 2}}

%\subtitle{}

% - Use the \inst{?} command only if the authors have different
%   affiliation.
%\author{F.~Author\inst{1} \and S.~Another\inst{2}}
\author{Yawei Li} 

% - Use the \inst command only if there are several affiliations.
% - Keep it simple, no one is interested in your street address.
\institute[LMU]
{
\\
  \texttt{yawei.li@stat.uni-muenchen.de}
}

\date{February 4, 2025}


% This is only inserted into the PDF information catalog. Can be left
% out.
\subject{Subject}



% If you have a file called "university-logo-filename.xxx", where xxx
% is a graphic format that can be processed by latex or pdflatex,
% resp., then you can add a logo as follows:

% \pgfdeclareimage[height=0.5cm]{university-logo}{university-logo-filename}
% \logo{\pgfuseimage{university-logo}}



% Delete this, if you do not want the table of contents to pop up at
% the beginning of each subsection:
%\AtBeginSubsection[]
%{
%\begin{frame}<beamer>
%\frametitle{Outline}
%\tableofcontents[currentsection,currentsubsection]
%\end{frame}
%}

% If you wish to uncover everything in a step-wise fashion, uncomment
% the following command:

%\beamerdefaultoverlayspecification{<+->}

\begin{document}



\begin{frame}[noframenumbering, plain]
\titlepage

\end{frame}

\begin{frame}{Exercise 1: Gaussian Posterior Process}
Assume your data follows the following law:
\begin{align*}
	\yv = \fv + \epsv, \quad \epsv \sim \N(\zero, \sigma^2 \id),
\end{align*}
with $\fv = \fx \in \R^n$ being a realization of a Gaussian process (GP), for which we a priori assume 
\begin{equation*}
	\fx \sim \gp.
\end{equation*}
$\xv$ here only consists of 1 feature that is observed for $n$ data points.

(a) Derive / define the prior distribution of $\fv$.
\end{frame}

\begin{frame}{1 (a): Prior Distribution of $\fv$}
	\begin{itemize}
		\item $\fv \sim \N(\mv, \Kmat)$.
		\item $\mv = m (\xv)$.
		\item $\kij = \kxij$. 
		\item NB: Note the (in-)finite Gaussian property of a GP.
	\end{itemize}
\end{frame}

\begin{frame}{Exercise 1 (b)}
	(b) Derive the posterior distribution of $\fv | \yv$.
\end{frame}

\begin{frame}{1 (b): Likelihood and Prior}
\small
The likelihood is:
	\begin{equation*}
		\begin{aligned}
			p(\yv | \fv)
			&\propto \exp \left( -\frac{1}{2} (\yv - \fv)^\top (\sigma^2 \id)^{-1} (\yv - \fv) \right).
		\end{aligned}
	\end{equation*}
The prior is:
	\begin{align*}
		p(\fv) \propto \exp\left( - \frac{1}{2}(\fv - \mv)^\top \Kmat^{-1} (\fv - \mv) \right).
	\end{align*}
\end{frame}

\begin{frame}{1 (b): Posterior}
\small
	So the posterior is:
\begin{align*}
	p(\fv | \yv) & \propto p(\yv | \fv) \cdot p(\fv) \\
	&\propto \exp\left( -\frac{1}{2} (\yv - \fv)^\top (\sigma^2 \id)^{-1} (\yv - \fv) \right) \cdot \exp\left( - \frac{1}{2}(\fv - \mv)^\top \Kmat^{-1} (\fv - \mv) \right) \\
	&\propto \exp \left( -\frac{1}{2} \left(  \fv^\top \ssqiinv\fv - 2 \fv^\top \ssqiinv \yv + \fv^\top \kinv \fv - 2 \fv^\top \kinv \mv \right) \right) \\
	&\propto \exp \left( -\frac{1}{2} \left( \fv^\top (\ssqiinv + \kinv) \fv \  -2\cdot  \fv^\top (\ssqiinv \yv + \kinv \mv) \right) \right) \\
	& \propto \exp \bigg( -\frac{1}{2}\fv^\top \underbrace{(\ssqiinv + \kinv)}_{=: \kpinv} \fv \ + \ \fv^\top \underbrace{(\ssqiinv \yv + \kinv \mv)}_{=:\tfv}  \bigg) \\
	&\propto \exp \left( - \frac{1}{2} \fv^\top \kpinv \fv + \fv^\top \tfv \right)
\end{align*}
\end{frame}

\begin{frame}{1 (b): Complete the Square}
\small
	\begin{align*}
		p(\fv | \yv) &\propto \exp \left( - \frac{1}{2} \fv^\top \kpinv \fv + \fv^\top \tfv \right) \\
	\end{align*}
	Recall the technique of \textbf{completing the square}: 
	\begin{itemize}
		\item Scalar: $a x^2 + b x + c = a(x + \frac{b}{2a})^2 + (c - \frac{b^2}{4a})$
		\item Matrix / vector: $\xv^\top \Amat \xv + \xv^\top \bv +c = (\xv + \frac{1}{2} \Amat^{-1} \bv)^\top \Amat (\xv + \frac{1}{2} \Amat^{-1} \bv) + (c - \frac{1}{4} \bv^\top \Amat^{-1} \bv)$ for a \textbf{symmetric} matrix $\Amat$.
	\end{itemize}
	In our case: $\Amat = -\frac{1}{2} \kpinv$, and $\bv = \tfv$, and $c=0$ or an arbitary const. So, 
	$$\xv + \frac{1}{2} \Amat^{-1} \bv = \fv + \frac{1}{2}(-\frac{1}{2} \kpinv)^{-1} \tfv = \fv - \Kpost \tfv$$
	
	and we omit $c - \frac{1}{4} \bv^\top \Amat^{-1}\Amat \bv $ because
	\begin{itemize}
		\item It is a constant w.r.t. $\fv$.
		\item the $\exp$ and $\propto$ operators.
	\end{itemize}
\end{frame}

\begin{frame}{1 (b): Complete the Square}
	\begin{align*}
		p(\fv | \yv) &\propto \exp \bigg( (\fv - \Kpost \tfv )^\top \left( -\frac{1}{2} \kpinv \right)^{-1} (\fv - \underbrace{\Kpost \tfv}_{:= \fpost})\bigg) \\ &\propto \exp \bigg( (\fv - \fpost)^\top \left( -\frac{1}{2} \kpinv \right)^{-1} (\fv - \fpost)\bigg)
	\end{align*}
	Hence,
	\begin{align*}
		\fv | \yv \sim \N(\fpost, \Kpost).
	\end{align*}
\end{frame}

\begin{frame}{Exercise 1 (c)}
	(c) Derive the posterior predictive distribution $\ys | \xs, \xv, \yv$ for a new sample $\xs$ from the sample data-generating process.
\end{frame}

\begin{frame}{1 (c): Derive Predictive Posterior from Joint Distribution}
	Na\"ively, we can compute 
	\begin{align*}
		p(\ys | \xs, \xv, \yv) =\int p(\ys | \xs, \xv, \yv, \fv) \cdot p(\fv | \yv, \xv) \mathsf{d} \fv.
	\end{align*}
	This is cumbersome. Alternative, we can use the fact that 
	\begin{align*}
		\text{if } \begin{pmatrix}
			\av \\
			\bv \\
		\end{pmatrix} 
		\sim \N 
		\left( 
			\begin{pmatrix}
				\muv_{a} \\
				\muv_{b} \\
			\end{pmatrix}, 
			\begin{pmatrix}
				\Sigma_{aa} & \Sigma_{ab} \\
				\Sigma_{ab}^\top & \Sigma_{bb} \\
			\end{pmatrix}
		\right), 
		\ \text{then } \quad  \av | \bv \sim \N(\muv_{a|b}, \Sigma_{a|b})
	\end{align*}
	where 
	\begin{align*}
		\muv_{a|b} &= \muv_a + \Sigma_{ab} \Sigma_{bb}^{-1}(\bv - \muv_b) \\
		\Sigma_{a|b} &= \Sigma_{aa} - \Sigma_{ab} \Sigma_{bb}^{-1} \Sigma_{ba}
	\end{align*}
\end{frame}

\begin{frame}{1 (c) Derive Predictive Posterior from Joint Distribution}
	The joint distribution of $(\yv, \ys)$:
	
	\textbf{Note: here is $\yv$ instead of $\fv$, and $\yv = \fv + \epsv$.  So we have $\sigma^2$ in the in the cov. matrix.}
	\begin{align*}
		\begin{pmatrix}
			\yv \\
			\ys 
		\end{pmatrix}
		\sim \N \left(
		\begin{pmatrix}
			\mv \\
			\ms
		\end{pmatrix},
		\begin{pmatrix}
			\Kmat + \sigma^2 \id & \Kstar \\
			\Kstar^\top  & \Kstarstar + \sigma^2 
		\end{pmatrix}
		\right),
	\end{align*}
	
	Therefore, the conditional distribution $\ys | \xs, \xv, \yv $ is also a Gaussian:
	\begin{align*}
		\ys | \xs, \xv, \yv \sim \N(\ms + \Kstar^\top(\Kmat + \sigma^2 \id)^{-1}(\yv -\mv), \Kstarstar + \sigma^2 - \Kstar^\top (\Kmat +\sigma^2 \id)^{-1} \Kstar).
	\end{align*}
\end{frame}

\begin{frame}{Exercise 1 (d)}
	Implement the GP with squared exponential kernel, zero mean function and $\ell = 1$ from scratch for $n=2$ observations $(\yv, \xv)$. Do this as efficiently as possible by explicitly calculating all expensive computations by hand. Do the same for the posterior predictive distribution of $\ys$. Test your implementation using simulated data.
	
	\textbf{Show the standard solution.}
\end{frame}

\end{document}
