%% LaTeX Beamer presentation template (requires beamer package)
%% see http://bitbucket.org/rivanvx/beamer/wiki/Home
%% idea contributed by H. Turgut Uyar
%% template based on a template by Till Tantau
%% this template is still evolving - it might differ in future releases!

%% Template edited by Panagiotis Adamopoulos {padamopo}@stern.nyu.edu

\documentclass[aspectratio=169]{beamer}
 
\mode<presentation>
{
\usetheme{NYU}

\setbeamercovered{transparent}
}

\usepackage[english]{babel}
\usepackage[latin1]{inputenc}
\usepackage[scaled=.90]{helvet}
\usepackage{courier}
\usepackage[T1]{fontenc}
\usepackage{comment}
%usepackage{appendixnumberbeamer}
\usepackage{amsmath}
\usepackage{bm}
\usepackage{pgfpages}
% citations
\usepackage{natbib}
\bibpunct{(}{)}{;}{a}{,}{,}
\def\citeapos#1{\citeauthor{#1}'s (\citeyear{#1})}
\renewcommand{\bibsection}{\subsubsection*{\bibname } }

\input{latex-math/basic-math.tex}
\input{latex-math/basic-ml.tex}

\newcommand{\fyx}{f_y(\xv)}


\title[]{\textbf{Exercise of Supervised Learning: \\ Multiclass Classification}}

%\subtitle{}

% - Use the \inst{?} command only if the authors have different
%   affiliation.
%\author{F.~Author\inst{1} \and S.~Another\inst{2}}
\author{Yawei Li} 

% - Use the \inst command only if there are several affiliations.
% - Keep it simple, no one is interested in your street address.
\institute[LMU]
{
\\
  \texttt{yawei.li@stat.uni-muenchen.de}
}

\date{November 12, 2024}


% This is only inserted into the PDF information catalog. Can be left
% out.
\subject{Subject}



% If you have a file called "university-logo-filename.xxx", where xxx
% is a graphic format that can be processed by latex or pdflatex,
% resp., then you can add a logo as follows:

% \pgfdeclareimage[height=0.5cm]{university-logo}{university-logo-filename}
% \logo{\pgfuseimage{university-logo}}



% Delete this, if you do not want the table of contents to pop up at
% the beginning of each subsection:
%\AtBeginSubsection[]
%{
%\begin{frame}<beamer>
%\frametitle{Outline}
%\tableofcontents[currentsection,currentsubsection]
%\end{frame}
%}

% If you wish to uncover everything in a step-wise fashion, uncomment
% the following command:

%\beamerdefaultoverlayspecification{<+->}


\begin{document}



\begin{frame}[noframenumbering, plain]
\titlepage

\end{frame}


\begin{frame}{Exercise 4: Multiclass Hinge Loss}
	\small
	Consider the multiclass classification scenario consisting of a feature space $\Xspace$ and a label space $\Yspace = \{1, \ldots, g\}$ with $g \geq 2 $ classes. Moreover, we consider the hypothesis space of models based on $g$ discriminant/scoring functions:
	$$\Hspace = \{ f = (f_1, \ldots, f_g)^T: \Xspace \to \mathbb{R}^g | f_k: \Xspace \to \mathbb{R},  \ \forall k \in \Yspace \}.$$
	A model $f$ in $\Hspace$ is used to make a prediction by means of transforming the scores into classes by choosing the class with the maximum score:
	\begin{align}
		h(\xv) = \argmax_{k \in \{1, \ldots, g \}} f_k(\xv).
		\label{eq:1}
	\end{align} 
	The multiclass hinge loss is defined by $$L(y, \fx) = \max_k (\fkx - f_y(\xv) + \id_{y \neq k}). \qquad \rhd$$
	(a) Show that 0-1-loss for a predictor $h$ as in (\ref{eq:1}) based on a model $f \in \Hspace$ is at most the multiclass hinge loss for $f$ i.e., $$L_{0-1}(y, \hx) = \id_{y\neq \hx} \leq L(y, \fx).$$
\end{frame}

\begin{frame}{Solution to Question (a)}
	\small
	There are two cases: $y = \argmax_{k} f_k(\xv)$ or $y \neq \argmax_k \fkx$.
	\vspace{10pt}
	
	\uncover<2->{
		If $y = \argmax_{k} f_k(\xv)$, the 0-1-loss is 
		\begin{align*}
			L_{0-1}(y, \hx) = 0
		\end{align*}
		because $y = \argmax_k f_k(x)$. That is, the prediction via argmax is correct. 
	}
	
	\uncover<3->{
		Now we analyzie $L(y, \fx)$. We need to evaluate $\fkx - f_y(\xv) + \id_{y \neq k}$ for all $k \in [1, 2, \ldots, g]$, then we take the maximal number. Note that for $k = y$, we have 
		$$\fkx - f_y(\xv) + \id_{y \neq k} = f_y(\xv) - f_y(\xv) + 0 = 0$$
		So, it means the max over all $k$-s yields 0 or some larger number. Formally,  
		\begin{align*}
			L(y, \fx) 
			&= \max_k (\fkx - f_y(\xv) + \id_{y \neq k}) \\
			&\geq f_y(\xv) - f_y(\xv) + 0 = 0 \
		\end{align*}
		
		So $L(y, \fx) \geq  L_{0, 1} (y, \hx)$ holds.
	}
	
\end{frame}

\begin{frame}{Solution to Question (a): Continued}
	\small
	
	If $y \neq \argmax_{k} f_k(\xv)$, then
	\begin{align*}
		L(y, \fx) 
			&= \max_k (\underbrace{\fkx - f_y(\xv)}_{> 0} + \underbrace{\id_{y \neq k}}_{= 1}) \\
			&> 1.
	\end{align*}
	\uncover<2->{
		The 0-1-loss is 
		\begin{align*}
			L_{0-1}(y, \hx) = 1
		\end{align*}
		because $y \neq \argmax_k f_k(x)$. That is, the prediction via argmax is incorrect. So $L(y, \fx) > L_{0, 1} (y, \hx)$.
	}
	
	\vspace{10pt}
	\uncover<3->{
		Combining the two cases, we have proved that $$L_{0-1}(y, \hx) = \id_{y\neq \hx} \leq L(y, \fx).$$
	}
\end{frame}

\begin{frame}{Question (b)}
	(b) Verify that the multiclass hinge loss of $f \in \Hspace$ on a data point $(\xv, y) \in \Xspace \times \Yspace$ is bounded from above by $\sum_{k\neq y} \max \{0, 1 + \fkx - f_y(\xv) \}$.
	\vspace{10pt}
	\emph{Hint}: Note that this upper bound is sometimes referred to as the multiclass hinge loss.
\end{frame}

\begin{frame}{Solution to Question (b)}
	\small
	\uncover<1->{	
	Let's assume $k^* = \argmax_k \fkx$. Note that $k^*$ can be $y$ or some other classes.
	}
	\vspace{10pt}
	
	\uncover<2->{
	\textbf{Case 1}: If $k^* = y$, then $L(y, \fx) = f_{k^*}(\xv) - f_y(\xv) + 0 =0$. So,  $L(y, \fx) \leq \sum_{k \neq y} \max \{0, 1 + \fkx - \fyx\}$. (Sum of non-neg. number is also non-neg.)
		So we have finished the proof for this case.
	}
	\vspace{10pt}
	
	\uncover<3->{
	\textbf{Case 2}: If $k^* \neq y$, then $L(y, \fx) = f_{k^*}(\xv) - f_y(\xv) + 1 > 1$.  (Result from (a)).

	So in this case, we can write
	\begin{align*}
		L(y, \fx) &= \underbrace{f_{k^*}(\xv) - f_y(\xv) + 1}_{\geq 0} \\
		&= \max\{0, f_{k^*}(\xv) - f_y(\xv) + 1 \}.
	\end{align*}
	}
	
	\uncover<4->{
	But our goal is to compare $L(y, \fx)$ with $\sum_{k \neq y} \max \{0, 1 + \fkx - \fyx\}$. So we need to add more terms of other $k$ with $k\neq k^*$. For other classes, we can apply the same fact that $\max \{0, 1 + \fkx - \fyx\} \geq 0$, as $\max\{0, \cdot\}$ always gives a non-neg. number.
	}
\end{frame}

\begin{frame}{Solution to Question (b): Continue}
	\small
	\uncover<2->{
		For Case 2 where $k^* \neq y$, it follows that
		\begin{align*}
			L(y, \fx) 
			&= \underbrace{f_{k^*}(\xv) - \fyx + 1}_{ \geq 0} \\
			&= \max\{0, f_{k^*}(\xv) - \fyx + 1\} \\
			&\leq \max\{0, f_{k^*}(\xv) - \fyx + 1\} + \underbrace{\max\{0, f_1(\xv) - \fyx + 1\} + \ldots + \max\{0, f_g(\xv) - \fyx + 1\}}_{\forall j \in \Yspace \ \text{and } j \neq y \ \text{and } j \neq k^*} \\
			&= \sum_{k \neq y} \max \{0, 1 + \fkx - \fyx\}
		\end{align*}
		So, combining the two cases, $\sum_{k \neq y} \max \{0, 1 + \fkx - \fyx\}$ is an upper bound for $L(y, \fx)$.
	}
\end{frame}

\begin{frame}{Question (c)}
	In the case of binary classification, i.e., $g=2$ and $\Yspace = \{-1, +1\}$, we use a single discriminant model $\fx = f_1(\xv) - f_{-1}(\xv)$ based on two scoring functions: $f_1, f_{-1}: \Xspace \to \mathbb{R}$ for the prediction by means of $\hx = \mathrm{sgn}(\fx)$. Here, $f_1$ is the score for the positive class and $f_{-1}$ is the score for the negative class. Show that the upper bound in (b) coincide with the binary hinge loss $L(y, \fx) = \max \{0, 1 - y \fx \}$.
\end{frame}

\begin{frame}{Solution to Question (c)}
	Case 1: $y= + 1$. In this case,
	\begin{align*}
		\sum_{k \neq y} \max \{0, 1 + \fkx - \fyx\}
		&= \sum_{k \neq + 1} \max \{ 0, \fkx - f_1(\xv) + 1 \} \\
		&= \max \{ 0, \underbrace{f_{-1}(\xv) - f_1(\xv)}_{:= - \fx} + 1 \} \\
		&= \max \{ 0, 1 - \fx \} \\
		&= \max \{ 0, 1 - y \cdot \fx \}
	\end{align*}
	So the equation holds in this case.
\end{frame}

\begin{frame}{Solution to Question (c): Continued}
	Case 2: $y = -1$. In this case,
	\begin{align*}
		\sum_{k \neq y} \max \{0, 1 + \fkx - \fyx \}
		&= \sum_{k \neq -1} \max \{ 0, \fkx - f_{-1}(\xv) + 1 \} \\
		&= \max \{ 0, f_1(\xv) - f_{-1}(\xv) + 1 \} \\
		&= \max \{0, 1 + \fx \} \\
		&= \max \{0, 1 - y\cdot \fx \}
	\end{align*}
	
Therefore, it is proven that the equation holds in two cases.
\end{frame}

\begin{frame}{Question (d)}
	Recall the statement of the lecture regarding the binary hinge loss:
	\vspace{10pt}
	
	\emph{``... the hinge loss only equals zero for a margin $\geq 1$ encouraging confident (correct) predictions.''}
	\vspace{10pt}
	
	Can we say something similar for the altenative multiclass hinge loss in (b)?
	
	Hint: multiclass hinge loss: $\sum_{k\neq y} \max \{0, 1 + \fkx - f_y(\xv) \}$, and all $\max\{0, \cdot\}$ terms are non-negative.
\end{frame}

\begin{frame}{Solution to Question (d)}
	Yes, we can say that \emph{it is only zero if all the $g-1$ margins are greater than 1.}
	\begin{itemize}
		\item<2-> Margins: $m_{y, k}(\xv) = \fyx - \fkx$, where $k \in \Yspace \setminus \{y\}$.
		\item<3-> Mathematically: $ m_{y, k}(\xv) \geq 1 \ \forall k \neq y  \Leftrightarrow \sum_{k \neq y} \max \{0, 1 + \fkx - \fyx \} = 0$.
	\end{itemize}
	
	\vspace{10pt}
	\uncover<4->{
		Proof: 
		\begin{align*}
			m_{y, k}(\xv) \geq 1 \ \forall k \neq y  
			&\Rightarrow \fyx - \fkx  \geq 1 \ \forall k \neq y \\
			&\Rightarrow \fkx - \fyx \leq -1 \ \forall k \neq y \\
			&\Rightarrow \max \{0, 1 + \fkx - \fyx \} = 0 \ \forall k \neq y \\
			&\Rightarrow \sum_{k \neq y} \max \{0, 1 + \fkx - \fyx \} = 0
		\end{align*}
	}
\end{frame}

\begin{frame}{Question (e) and Solution to (e)}
	\textbf{Show the standard solution.}
\end{frame}

\end{document}
