%% LaTeX Beamer presentation template (requires beamer package)
%% see http://bitbucket.org/rivanvx/beamer/wiki/Home
%% idea contributed by H. Turgut Uyar
%% template based on a template by Till Tantau
%% this template is still evolving - it might differ in future releases!

%% Template edited by Panagiotis Adamopoulos {padamopo}@stern.nyu.edu

\documentclass[aspectratio=169]{beamer}
 
\mode<presentation>
{
\usetheme{NYU}

\setbeamercovered{transparent}
}

\usepackage[english]{babel}
\usepackage[latin1]{inputenc}
\usepackage[scaled=.90]{helvet}
\usepackage{courier}
\usepackage[T1]{fontenc}
\usepackage{comment}
%usepackage{appendixnumberbeamer}
\usepackage{amsmath}
\usepackage{bm}
\usepackage{pgfpages}
% citations
\usepackage{natbib}
\bibpunct{(}{)}{;}{a}{,}{,}
\def\citeapos#1{\citeauthor{#1}'s (\citeyear{#1})}
\renewcommand{\bibsection}{\subsubsection*{\bibname } }

\input{latex-math/basic-math.tex}
\input{latex-math/basic-ml.tex}

\newcommand{\Ex}{\mathbb{E}_{\xv}}
\newcommand{\Eyx}{\mathbb{E}_{y|\xv}}
\newcommand{\Prob}{\mathbb{P}}
\newcommand{\my}{m(y)}
\newcommand{\myi}{m(\yi)}
\newcommand{\mfx}{m(f(\xv))}
\newcommand{\mfxi}{m(f(\xv^{(i)}))}
\newcommand{\zi}{z^{(i)}}
\newcommand{\mfruexi}{m(\ftrue(\xv^{(i)}))}

\title[]{\textbf{Supervised Learning: Exercise 3}}

%\subtitle{}

% - Use the \inst{?} command only if the authors have different
%   affiliation.
%\author{F.~Author\inst{1} \and S.~Another\inst{2}}
\author{Yawei Li} 

% - Use the \inst command only if there are several affiliations.
% - Keep it simple, no one is interested in your street address.
\institute[LMU]
{
\\
  \texttt{yawei.li@stat.uni-muenchen.de}
}

\date{November 17, 2023}


% This is only inserted into the PDF information catalog. Can be left
% out.
\subject{Subject}



% If you have a file called "university-logo-filename.xxx", where xxx
% is a graphic format that can be processed by latex or pdflatex,
% resp., then you can add a logo as follows:

% \pgfdeclareimage[height=0.5cm]{university-logo}{university-logo-filename}
% \logo{\pgfuseimage{university-logo}}



% Delete this, if you do not want the table of contents to pop up at
% the beginning of each subsection:
%\AtBeginSubsection[]
%{
%\begin{frame}<beamer>
%\frametitle{Outline}
%\tableofcontents[currentsection,currentsubsection]
%\end{frame}
%}

% If you wish to uncover everything in a step-wise fashion, uncomment
% the following command:

%\beamerdefaultoverlayspecification{<+->}

\begin{document}



\begin{frame}[noframenumbering, plain]
\titlepage

\end{frame}


\begin{frame}{Exercise 3: Connection between between MLE and ERM}
\small
	Suppose we are facing a regression task, i.e., $\Yspace = \mathbb{R}$, and the feature space $\Xspace \subseteq \mathbb{R}^p$. Let us assume that the relationship between the features and labels is specified by 
	$$y = m^{-1}(m(\ftrue(\xv)) + \eps), \qquad \lhd$$
	where $m: \mathbb{R} \to \mathbb{R}$ is a continuous strictly monotone function with $m^{-1}$ being its inverse function, and the errors are Gaussian, i.e., $\eps \sim \mathcal{N}(0, \sigma^2)$. In particular, for the data points $(\xv^{(1)}, y^{(1)}, \ldots, \xv^{(n)}, y^{(n)})$ it holds that
	$$\yi = m^{-1}(m(\ftrue(\xv^{(i)})) + \epsi),$$
	where $\eps^{(1)}, \ldots, \eps^{(n)}$ are i.i.d. with distribution $\mathcal{N}(0, \eps^2)$.
	
	\textbf{Disclaimer:} We assume in the following that $\my$ and $\mfx$ is well-defined for any $y \in \Yspace, f \in \Hspace$ and $\xv \in \Xspace$.
	
	(a) How can we transform the labels $y^{(1)}, \ldots, y^{(n)}$ to ``new'' labels $z^{(1)}, \ldots, z^{(n)}$ such that $ z^{(i)} \ | \ \xv$ is normally distributed? What are the parameters of this normal distribution?
	
\end{frame}

\begin{frame}{Solution: Question (a)}
\uncover<1->{
	We can use 
	$$z^{(i)} = m(y^{(i)}).$$

	We know that $ \yi = m^{-1}(\mfruexi + \epsi)$, therefore $	\myi = \mfruexi + \epsi$.
}
\uncover<2->{
	Hence, 
	$$\zi = \mfruexi + \epsi, \quad \text{with } \eps^{(i)} \sim \mathcal{N}(0, \sigma^2). $$
	Therefore, $\zi \ | \ \xv^{(i)}$ is normally distributed, i.e., $$\zi \ | \ \xv^{(i)} \sim  \mathcal{N}(\mfruexi, \sigma^2)$$.
}
\end{frame}

\begin{frame}{Question (b)}
	Assume that the hypothesis space is 
	$$\Hspace = \{ f(\cdot, \thetab): \Xspace \to \mathbb{R} | f(\cdot | \thetab) \text{ belongs to a certain functional family parameterized by } \thetab \in \Theta \},$$
	where $\thetab = (\theta_1, \theta_2, \ldots, \theta_d))$ is a parameter vector, which is an element of a \textbf{parameter space} $\Theta$. Based on your findings (a), establish a relationship between minimizing the negative log-likelihood for $(\xv^{(1)}, z^{(1)}), 
	\ldots (\xv^{(n)}, z^{(n)})$ and empirical loss minimizing over $\Hspace$ of the generalized L2-loss function of Exercise sheet 1, i.e., $L(y, \fx)) = (\my - \mfx)^2$.
\end{frame}

\begin{frame}{Solution to Question (b)}
	\small
	The likelihood for  $(\xv^{(1)}, z^{(1)}), \ldots (\xv^{(n)}, z^{(n)})$ is 
	\begin{align*}
		\only<1->{
			\mathcal{L}(\thetab) 
			&= \prodin p \left( \zi \ | \ f(\xi \ | \ \thetab), \sigma^2 \right)  \\}
		\only<2->{&\propto \exp \left( -\frac{1}{2\sigma^2} \sumin \left[ \zi - m(f(\xi \ | \ \thetab)) \right]^2 \right).}
	\end{align*}
	\uncover<3->{
		So the negative log-likelihood (NLL) is 
	}
	\begin{align*}
		\only<4->{- \ell(\thetab) &= - \log(\mathcal{L}(\thetab))\\}
		\only<5->{&\propto \sumin  \left[ \zi - m(f(\xi \ | \ \thetab)) \right]^2 \\}
		\only<6->{&= \sumin  \left[ \myi - m(f(\xi \ | \ \thetab)) \right]^2 }
	\end{align*}
	
	\uncover<7->{Therefore, the NLL is propotional to the empirical risk of $f(\cdot | \thetab)$ w.r.t. the generalized L2-Loss.}
\end{frame}

\begin{frame}{Question (c)}
	(c) In many practical applications, one often observed statisical property is that the label $y$ given $\xv$ follows a \emph{log-normal distribution}. Note that we can obtain such a relationship by using $m(x) = \log(x)$ above. In the following we want to consider the conjecture of James D. Forbes, who conjectured in the year 1857 that the relationship between the air pressure $y$ and the boiling point of water $x$ is given by $$y = \theta_1 \exp(\theta_2 x + \eps),$$ for some specific values $\theta_1 \in \mathbb{R}_+, \theta_2 \in \mathbb{R}$ and some error terms $\eps$ (of course, we assume that this error term is stochastic and normally distributed).
	
	What would be a suitable hypothesis space $\Hspace$ if this conjecture holds?
\end{frame}

\begin{frame}{Solution to Question (c)}
	\small
	\begin{itemize}
		\item<1-> The goal is to introduce a proper form for $f(\cdot | \thetab)$ so that $y = m^{-1}(m(f(x)) + \eps)$.
		\item<2-> We use the transformation $m(\cdot) = \log(\cdot)$. So $m^{-1}(\cdot) = \exp(\cdot)$. 
		\item<3-> The Forbes' conjectured model $y = \theta_1 \exp(\theta_2 x + \eps)$ can be written as 
			\begin{align*}
				\only<4->{y &= \theta_1 \exp(\theta_2 x + \eps) \\}
				\only<5->{&= \exp(\log(\theta_1 \cdot \exp(\theta_2 x + \eps))) \\}
				\only<6->{&= \exp[\log \theta_1 + \log(\exp(\theta_2 x)) + \log (\exp(\eps))] \\}
				\only<7->{&= \exp[\log \theta_1 + \log(\exp(\theta_2 x)) + \eps] \\}
				\only<8->{&= \exp[\log (\theta_1 \cdot \exp(\theta_2 x)) + \eps] \\}
				\only<9->{&= m^{-1}(m(\theta_1 \cdot \exp(\theta_2 x)) + \eps).}
			\end{align*}
			\uncover<10->{
			Therefore, $f(x \ | \ \thetab) = \theta_1 \exp(\theta_2 x)$ is a suitable functional form for the hypothesis. So 
			$$\Hspace = \{ f(x \ | \ \thetab) = \theta_1 \exp(\theta_2 x) \ | \ \thetab \in \Theta \}.$$
			(The standard solution uses $\Xspace = {1} \times \mathbb{R}$, i.e., $\xv = (x1, x2)^T = (1, x_2)^T$.)}
	\end{itemize}
\end{frame}

\begin{frame}{Solution to Question (c): Continued}


	\textbf{Show the code in the standard solution.}

	
\end{frame}

\end{document}
