%% LaTeX Beamer presentation template (requires beamer package)
%% see http://bitbucket.org/rivanvx/beamer/wiki/Home
%% idea contributed by H. Turgut Uyar
%% template based on a template by Till Tantau
%% this template is still evolving - it might differ in future releases!

%% Template edited by Panagiotis Adamopoulos {padamopo}@stern.nyu.edu

\documentclass[aspectratio=169]{beamer}
 
\mode<presentation>
{
\usetheme{NYU}

\setbeamercovered{transparent}
}

\usepackage[english]{babel}
\usepackage[latin1]{inputenc}
\usepackage[scaled=.90]{helvet}
\usepackage{courier}
\usepackage[T1]{fontenc}
\usepackage{comment}
%usepackage{appendixnumberbeamer}
\usepackage{amsmath}
\usepackage{bm}
\usepackage{pgfpages}
% citations
\usepackage{natbib}
\bibpunct{(}{)}{;}{a}{,}{,}
\def\citeapos#1{\citeauthor{#1}'s (\citeyear{#1})}
\renewcommand{\bibsection}{\subsubsection*{\bibname } }

\input{latex-math/basic-math.tex}
\input{latex-math/basic-ml.tex}
\input{latex-math/ml-gp.tex}

\newcommand{\Norm}{\mathcal{N}}
\newcommand{\sumip}{\sum_i^p}
\newcommand{\thetai}{\thetav_i}
\newcommand{\xvp}{\xv^\prime}
\newcommand{\kz}{k_0}
\newcommand{\ko}{k_1}
\newcommand{\kzxxp}{\kz(\xv, \xvp)}
\newcommand{\koxxp}{\ko(\xv, \xvp)}
\newcommand{\Imat}{\bm{I}}
\newcommand{\Kij}{K_{i,j}}
\newcommand{\vv}{\bm{v}}
\newcommand{\dv}{\bm{d}}


\title[]{\textbf{Exercise of Supervised Learning:\\Gaussian Processes 1}}

%\subtitle{}

% - Use the \inst{?} command only if the authors have different
%   affiliation.
%\author{F.~Author\inst{1} \and S.~Another\inst{2}}
\author{Yawei Li} 

% - Use the \inst command only if there are several affiliations.
% - Keep it simple, no one is interested in your street address.
\institute[LMU]
{
\\
  \texttt{yawei.li@stat.uni-muenchen.de}
}

\date{January 28, 2025}


% This is only inserted into the PDF information catalog. Can be left
% out.
%\subject{Subject}



% If you have a file called "university-logo-filename.xxx", where xxx
% is a graphic format that can be processed by latex or pdflatex,
% resp., then you can add a logo as follows:

% \pgfdeclareimage[height=0.5cm]{university-logo}{university-logo-filename}
% \logo{\pgfuseimage{university-logo}}



% Delete this, if you do not want the table of contents to pop up at
% the beginning of each subsection:
%\AtBeginSubsection[]
%{
%\begin{frame}<beamer>
%\frametitle{Outline}
%\tableofcontents[currentsection,currentsubsection]
%\end{frame}
%}

% If you wish to uncover everything in a step-wise fashion, uncomment
% the following command:

%\beamerdefaultoverlayspecification{<+->}

\begin{document}



\begin{frame}[noframenumbering, plain]
\titlepage

\end{frame}

\begin{frame}{Exercise 1: Bayesian Linear Model}
\small
In the Bayesian linear model, we assume that the data follows the following law:
\begin{equation*}
	y = \fx + \eps = \thetav^\top \xv + \eps,
\end{equation*}
where $\eps \sim \Norm(0, \sigma^2)$ and independent of $\xv$. On the data-level this corresponds to 
\begin{equation*}
	\yi = \fxi + \epsi = \thetav^\top \xi + \epsi, \quad \text{for } i \in [n],
\end{equation*}
where $\epsi \in \Norm(0, \sigma^2)$ are i.i.d. and all independent of $\xi$'s. In the Bayesian perspective it is assumed that the parameter vector $\thetav$ is stochastic and follows a distribution.
%
Assume we are interested in the so-called maximum a posteriori estimate of $\thetav$, which is defined by $$\thetavh = \argmax_{\thetav} p(\thetav | \Xmat, \yv).$$
%
(a) Show that if we choose a \textbf{uniform distribution} over the parameter vector $\thetav$ as the prior belief, i.e., $q(\thetav) \propto 1$, then the maximum a posteriori estimate coincides with the \textbf{empirical risk minimizer for the L2-loss} (over linear models). 	
\end{frame}

\begin{frame}{1 (a): Construct Posterior from Bayes' Rule}
	\begin{equation*}
		\begin{aligned}
			\only<1->{
			\underbrace{p(\thetav | \Xmat, \yv)}_{\text{posterior}} &= \frac{p(\thetav, \yv| \Xmat)}{p(\yv | \Xmat)} \\
			}
			\only<2->{
			&= \frac{\overbrace{p(\yv | \Xmat, \thetav)}^{\text{likelihood}} \overbrace{q(\thetav)}^{prior} }{\underbrace{p(\yv | \Xmat)}_{\text{marginal}}}
			}
		\end{aligned}
	\end{equation*}
	
	\begin{itemize}
			\item<3-> For a linear model, $\yv | \Xmat, \thetav \sim \Norm(\Xmat \thetav, \sigma^2 \Imat)$. Will be computed on the next slide.
			\item<4-> In 1(a), we choose a \textbf{uniform prior}, indicating $q(\thetav) \propto 1$.
			\item<5-> $p(\yv | \Xmat)$ does \textbf{not} depend on $\thetav$. $\leadsto$ Treated as  constant when maxing the posterior of $\thetav$.
	\end{itemize}
\end{frame}

\begin{frame}{1 (a): The Likelihood $p(\yv | \Xmat, \thetav)$}
	\begin{equation*}
		\begin{aligned}
			p(\yv | \Xmat, \thetav) &\propto \exp \left[ - \frac{1}{2\sigma^2} (\yv - \Xmat\thetav)^\top (\yv - \Xmat \thetav) \right] \\
			&= \exp \left[- \frac{||\yv - \Xmat \thetav||^2}{2 \sigma^2}\right] \\
			&= \exp \left[ -\frac{\sum_{i=1}^n(\yi - \thetav^\top \xi)^2}{2\sigma^2}  \right].
		\end{aligned}
	\end{equation*}
	\uncover<2->{
	
	In addition, recall that the prior $q(\thetav) \propto 1$ and we don't care the marginal $p(\yv | \Xmat)$.
	
	Now, we plug these information into $p(\thetav | \Xmat, \yv) = \frac{p(\yv | \Xmat, \thetav) q(\thetav)}{p(\yv|\Xmat)} \propto p(\yv | \Xmat, \thetav) q(\thetav)$.
	}
\end{frame}

\begin{frame}{1 (a): MAP Estimate}
	\small
	\begin{equation*}
			p(\thetav | \Xmat, \yv) \propto p(\yv | \Xmat, \thetav) \cdot q(\thetav) 
			\propto \exp \left[ -\frac{\sum_{i=1}^n(\yi - \thetav^\top \xi)^2}{2\sigma^2}  \right].
	\end{equation*}
	\uncover<2->{
	Now we compute the maximum a posterior estimate as:
	\begin{equation*}
		\begin{aligned}
			\thetavh &= \argmax_{\thetav} p(\thetav | \Xmat, \yv) \\
				&= \argmax_{\thetav} \log (p(\thetav | \Xmat, \yv)) \qquad \text{($\log$ is a monotone increasing func.)} \\
				&= \argmax_{\thetav}  -\frac{\sum_{i=1}^n(\yi - \thetav^\top \xi)^2}{2\sigma^2} \\
				&= \argmin_{\thetav} \frac{\sum_{i=1}^n(\yi - \thetav^\top \xi)^2}{2\sigma^2} \\
				&= \argmin_{\thetav} \sum_{i=1}^n (\yi - \thetav^\top \xi)^2.
		\end{aligned}
	\end{equation*}
Therefore, in 1 (a), maximum a posteriori estimate $\Leftrightarrow$ ERM for the L2-loss.
}
\end{frame}

\begin{frame}{Exercise 1(b)}
	Show that if we choose a \textbf{Gaussian distribution} over the parameter vectors $\thetav$ as the prior belief, i.e., 
	\begin{equation*}
		q(\thetav) \propto \exp \left[ - \frac{1}{2 \tau^2} \thetav^\top \thetav \right], \quad \tau > 0,
	\end{equation*}
	then the maximum a posteriori estimate coincides for a specific choice of $\tau$ with the \textbf{regularized} empirical risk miminizer for the L2-loss with L2 penalty (over the linear models), i.e., the Ridge regression.
\end{frame}

\begin{frame}{1 (b): Posterior with A Gaussian Prior}
\begin{equation*}
	\begin{aligned}
		p(\thetav | \Xmat, \yv) &\propto p(\yv | \Xmat, \thetav) q(\thetav) \\
		&\propto \exp \left[-\frac{\sum_{i=1}^n(\yi - \thetav^\top \xi)^2}{2\sigma^2}- \frac{1}{2 \tau^2} \thetav^\top \thetav \right] \\
		&\propto \exp \left[-\frac{\sum_{i=1}^n(\yi - \thetav^\top \xi)^2}{2\sigma^2} - \frac{||\thetav||_2^2}{2 \tau^2} \right].
	\end{aligned}
\end{equation*}
Next, we compute $\argmax_{\thetav} p(\thetav | \Xmat, \yv)$. That is, $\argmax_{\thetav} \log p(\thetav | \Xmat, \yv)$.
\end{frame}

\begin{frame}{1 (b): MAP Estimate}
\begin{equation*}
	\begin{aligned}
		\thetavh &= \argmax_{\thetav} \log p(\thetav | \Xmat, \yv) \\
		&= \argmax_{\thetav} -\frac{\sum_{i=1}^n(\yi - \thetav^\top \xi)^2}{2\sigma^2} - \frac{||\thetav||_2^2}{2 \tau^2} \\
		&= \argmin_{\thetav} \frac{\sum_{i=1}^n(\yi - \thetav^\top \xi)^2}{2\sigma^2} + \frac{||\thetav||_2^2}{2 \tau^2} \\
		&= \argmin_{\thetav} \sum_{i=1}^n(\yi - \thetav^\top \xi)^2 + \frac{\sigma^2}{\tau^2} ||\thetav||_2^2. \qquad \text{(Because $\times \sigma^2$ doesn't change the argmin)}
	\end{aligned}
\end{equation*}

We define $\lambda = \frac{\sigma^2}{\tau^2}$, then the maximum a posteriori $\Leftrightarrow$ L2-loss with L2 penalty.
	
\end{frame}

\begin{frame}{Exercise 1(c)}
	Show that if we choose a \textbf{Laplace distribution} over the parameter vectors $\thetav$ as the prior belief, i.e., 
	\begin{equation*}
		q(\thetav) \propto \exp \left[ - \frac{\sumip |\thetai|}{\tau} \right], \quad \tau > 0,
	\end{equation*}
	then the maximum a posteriori estimate coincides for a specific choice of $\tau$ with the regularized empirical risk minimizer for the L2-loss with L1 penalty (over the linear models), i.e., the Lasso regression.
\end{frame}

\begin{frame}{1 (c): Posterior with A Laplace Prior}
	\begin{equation*}
		\begin{aligned}
			p(\thetav | \Xmat, \yv) &\propto p(\yv | \Xmat, \thetav) q(\thetav) \\
			&\propto \exp \left[-\frac{\sum_{i=1}^n(\yi - \thetav^\top \xi)^2}{2\sigma^2}- \frac{||\thetav||_1}{\tau} \right]. \\
		\end{aligned}
	\end{equation*}
\end{frame}

\begin{frame}{1 (c): MAP Estimate}
	\begin{equation*}
		\begin{aligned}
			\thetavh &= \argmax_{\thetav} \log p(\thetav | \Xmat, \yv) \\
			&= \argmax_{\thetav} -\frac{\sum_{i=1}^n(\yi - \thetav^\top \xi)^2}{2\sigma^2} - \frac{||\thetav||_1}{\tau} \\
			&= \argmin_{\thetav} \frac{\sum_{i=1}^n(\yi - \thetav^\top \xi)^2}{2\sigma^2} + \frac{||\thetav||_1}{\tau} \\
			&= \argmin_{\thetav} \sum_{i=1}^n(\yi - \thetav^\top \xi)^2 + \frac{2 \sigma^2}{\tau} ||\thetav||_1 \qquad \text{(Because $\times \sigma^2$ doesn't change the argmin)}
		\end{aligned}
	\end{equation*}
	We define $\lambda = \frac{2 \sigma^2}{\tau}$, and then the MAP estimate $\Leftrightarrow$ L2-loss with L1 penalty.
\end{frame}

\begin{frame}{Exercise 2: Covariance Functions}
	\small
	Consider the commonly used covariance functions mentioned in the lecture slides: constant, linear, polynomial, squared exponential, Matern, exponential covariance functions.
	\vspace{1em}

	(a) Show that they are valid covariance functions. (\textbf{Proofs for Matern and exp. cov. functions are out of scope and omitted.}) You may use the following composition rules. In these rules we assume that $k_0(\cdot, \cdot)$ and $k_1(\cdot, \cdot)$ are valid covariance functions.
	\begin{enumerate}
		\item $\kxxp = \xv^\top \xvp$ is a valid covariance function;
		\item $\kxxp = c \cdot \kzxxp$ is a valid covariance function if $c \geq 0$ is constant.
		\item $\kxxp = \kzxxp + \koxxp$ is a valid covariance function;
		\item $\kxxp = \kzxxp \cdot \koxxp$ is a valid covariance function;
		\item $\kxxp = g(\kzxxp)$ is a valid cov. func. if $g$ is a polynomial function with \textbf{pos.} coefficients;
		\item $\kxxp = t(\xv) \cdot \kzxxp \cdot t(\xvp)$ is a valid covariance function, where $t$ is any function;
		\item $\kxxp = \exp(\koxxp)$ is a valid covariance function;
		\item $\kxxp = \xv^\top \Amat \xvp$ is a valid covariance function if $\Amat \succeq 0$.
	\end{enumerate}
\end{frame}

\begin{frame}{2 (a): Proof via Kernel Matrix}
	\small
	Construct of the kernel matrix $\Kmat \in \mathbb{R}^{n \times n}$ from $\D = \{\xv^{(1)}, \ldots, \xv^{(n)} \}$.
	
	Each element $\Kij = \kxxp$. In the current case $\kxxp = \sigma_0^2$ , we have $\Kij = \sigma_0^2$. In other words, 
	\begin{equation*}
		\Kmat = \begin{pmatrix}
			\sigma_0^2 & \ldots & \sigma_0^2 \\
			\vdots & \ddots & \vdots \\
			\sigma_0^2 & \ldots & \sigma_0^2 \\
		\end{pmatrix}
	\end{equation*}
	\textbf{Note:} kernel matrix $\Kmat$ is NOT kernel function $k(\cdot, \cdot)$. Don't claim ``$k(\cdot, \cdot)$ is P.S.D.'' in the exam. 
	 
	 Now, We need to prove that $\Kmat$ is P.S.D.
	\begin{enumerate}
		\item \textbf{Prove $\Kmat$ is symmetric}. 
		\item \textbf{Prove that $\forall \ \vv \in \mathbb{R}^n, \vv^\top \Kmat \vv \geq 0$.}
	\end{enumerate}
\end{frame}

\begin{frame}{2 (a): How to Prove That A Matrix Is P.S.D.}
\small
	\begin{enumerate}
		\item<1-> Since $\Kij = \sigma_0^2$ for all $i,j$, we have $\Kmat^\top = \Kmat$, thus $\Kmat$ is symmetric.
		\item<2-> For any $\vv = (v_1, v_2, \ldots, v_n)^\top$, we need to prove $ \vv^\top \Kmat \vv \geq 0$.
		\begin{enumerate}
			\item<3-> Naive way: First compute $(\Kmat \vv)$ and then $\vv^\top (\Kmat \vv)$. 
			\begin{align*}
				\Kmat \vv &= \sigma_0^2(\sum_i v_i, \sum_i v_i, \ldots, \sum_i v_i )^\top \\
				\vv^\top \Kmat \vv &= \sigma_0^2 [v_1 (\sum_i v_i) + v_2 (\sum_i v_i) + \ldots + v_n (\sum_i v_i)] = \sigma_0^2(v_1 + v_2 + \ldots + v_n) (\sum_i v_i) \\
				&= \sigma_0^2 (v_1 + \ldots + v_n) (\sum_i v_i) = \sigma_0^2 (\sum_i v_i) (\sum_i v_i) = \sigma_0^2 (\sum_i v_i)^2 \geq 0.
			\end{align*}
			\item<4-> Faster way: $\Kmat = \sigma_0^2\Imat \Imat^\top$, where $\Imat = (1, 1, \ldots, 1)^\top$. So, $\vv^\top \Kmat \vv = \sigma_0^2 \vv^\top \Imat \Imat^\top \vv = \sigma_0^2 (\Imat^\top \vv)^2 \geq 0$.
		\end{enumerate}
	\end{enumerate}
\end{frame}

\begin{frame}{2 (a): Proof via Transformed Feature Map}
	Alternatively, we can prove $\kxxp = \sigma_0^2$ is a valid cov. func. via writing $\kxxp$ as \textbf{an inner product of two transformed feature maps}.
	
	This requires to explicitly construct the feature map $\phi(\xv) \in \mathbb{R}^d$ for some $d$.
	
	In the current case, we can write 
	\begin{align*}
		\phi(\xv) = \sigma_0.
	\end{align*}
	So that 
	\begin{align*}
		\kxxp & = \sigma_0^2 \\
		&= \langle \sigma_0, \sigma_0 \rangle \\
		&= \langle \phi(\xv), \phi(\xvp) \rangle  \\
	\end{align*}
\end{frame}

\begin{frame}{2 (a): Proof of $\kxxp = \sigma_0^2 + \xv^\top \xvp$}
	\begin{align*}
		\kxxp = \underbrace{\sigma_0^2}_{:= k_0(\xv, \xvp)} + \underbrace{\xv^\top \xvp}_{:= k_1(\xv, \xvp)}
	\end{align*}
	\begin{enumerate}
		\item We have shown that $k_0$ is a valid cov. func.
		\item $k_1$ is a inner product $\langle \phi(\xv), \phi(\xvp) \rangle$, where $\phi(\xv) := \xv$. So, $k_1$ is a valid cov. func.
		\item Their sum $k_0 + k_1$ is also a cov. func.
	\end{enumerate}
\end{frame}

\begin{frame}{2 (a): Proof of $\kxxp = (\sigma_0^2 + \xv^\top \xv)^p$}
	
	We define $k_2(\xv, \xvp) =\sigma_0^2 + \xv^\top \xv$, so $k_3 = (\sigma_0^2 + \xv^\top \xv)^p = k_2^p$.
	
	\begin{enumerate}
		\item We have shown that $k_2$ is a cov. func.
		\item $k_3 = k_2^p$ is a polynomial of $k_2$ with only one $p$-order item $k_2^p$, and \textbf{the polynomial coefficient $1$ is positive}. So $k_3 = k_2^p$ is a cov. func.
	\end{enumerate}
\end{frame}

\begin{frame}{2 (a): Proof of $\kxxp = \exp\left(- \frac{||\xv - \xvp||_2^2}{2 \ell^2}\right)$}
\small
	We can write
	\begin{align*}
		\kxxp &= \exp\left(- \frac{||\xv - \xvp||_2^2}{2 \ell^2}\right) \\
		&=\exp \left( - \frac{\xv^\top \xv - 2 \xv^\top \xvp + \xv^{\prime \top} \xvp}{2 \ell^2} \right) \\
		&= \underbrace{\exp \left(-\frac{\xv^\top \xv}{2 \ell^2} \right)}_{:= t(\xv)}\cdot 
		\underbrace{\exp \left(\frac{\xv^\top \xvp}{\ell^2} \right)}_{:=k_4(\xv, \xvp)} \cdot \underbrace{\exp \left( - \frac{\xv^{\prime \top} \xvp}{2 \ell^2} \right)}_{:= t(\xvp)}.\\ 
	\end{align*}
	where we defined a function $t(\cdot)$.
	
	Furthermore, $\xv^\top \xvp$ is cov. func., so $\frac{\xv^\top \xvp}{\ell^2}$ is a kernel, so $\exp(\frac{\xv^\top \xvp}{\ell^2})$ is a cov. func. 
	
	Therefore, $\kxxp = t(\xv) \cdot k_4(\xv, \xvp) \cdot t(\xvp)$ is a cov. func.
\end{frame}

\begin{frame}{Exercise 2 (b)}
	(b): Are these covariance functions stationary or isotropic? Justify your answer.
\end{frame}

\begin{frame}{2 (b): Stationary and Isotropic}
	\begin{enumerate}
		\item $k(\cdot, \cdot)$ is stationary if $k(\xv, \xv + \dv) = k(\zero, \dv)$.
		\item $k(\cdot, \cdot)$ is isotropic if it is a function of $||\xv - \xvp||$.
	\end{enumerate}
\end{frame}

\begin{frame}{2 (b): Constant functions}
	\begin{enumerate}
		\item<1-> $\kxxp = \sigma_0^2$ is stationary since $k(\xv, \xv + \dv) = \sigma_0^2  = k(\zero, \dv)$.
		\item<2-> $\kxxp = \sigma_0^2$ is isotropic since $ \kxxp = \sigma_0^2||\xv - \xvp ||^0$.
	\end{enumerate}
\end{frame}

\begin{frame}{2 (b): $\kxxp = \sigma_0^2 + \xv^\top \xvp$}
	\begin{enumerate}
		\item<1-> $\kxxp = \sigma_0^2 + \xv^\top \xvp$ is NOT stationary, since
		\begin{align*}
			k(\xv, \xv + \dv) &= \sigma_0^2 + \xv^\top \xv + \xv^\top \dv \\
			k(\zero, \dv) &= \sigma_0^2.
		\end{align*}
		\item<2-> It is NOT isotropic, since it cannot be written as a func. of $|| \xv - \xvp||$.
	\end{enumerate}
\end{frame}

\begin{frame}{2 (b): Polynomial Cov. Func.}
	Similar to linear covariance functions, the polynomial covariance function is NOT stationary and NOT isotropic. (Prove this on your own.)
\end{frame}

\begin{frame}{2 (b): $\kxxp = \exp\left( - \frac{||\xv - \xvp||_2^2}{2 \ell^2}\right)$}
	\begin{enumerate}
		\item<1-> $\kxxp = \exp\left( - \frac{||\xv - \xvp||_2^2}{2 \ell^2}\right)$ is stationary, since $k(\xv, \xv + \dv) = \exp \left( - \frac{||\dv||_2^2}{2 \ell^2} \right) = k(\zero, \dv)$.
		\item<2-> It is isotropic, since it is a function of $||\xv - \xvp||$.
	\end{enumerate}
\end{frame}

\begin{frame}{2 (b): Matern and Exponential Cov. Func.}
	Similar to the argument of squared exponential conv. func. $\kxxp = \exp\left( - \frac{||\xv - \xvp||_2^2}{2 \ell^2}\right)$.
	\begin{enumerate}
		\item Matern cov. func. is stationary and isotropic.
		$$\kxxp = \frac{1}{2^\nu} \left( \frac{\sqrt{2 \nu}}{\ell} || \xv - \xvp|| \right)^\nu K_\nu \left( \frac{\sqrt{2 \nu}}{\ell} || \xv - \xvp|| \right)$$
		\item Exponential cov. func. is stationary and isotropic.
		$$\kxxp = \exp \left( - \frac{||\xv - \xvp ||}{\ell} \right)$$
	\end{enumerate}
\end{frame}

\end{document}
