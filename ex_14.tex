%% LaTeX Beamer presentation template (requires beamer package)
%% see http://bitbucket.org/rivanvx/beamer/wiki/Home
%% idea contributed by H. Turgut Uyar
%% template based on a template by Till Tantau
%% this template is still evolving - it might differ in future releases!

%% Template edited by Panagiotis Adamopoulos {padamopo}@stern.nyu.edu

\documentclass[aspectratio=169]{beamer}
 
\mode<presentation>
{
\usetheme{NYU}

\setbeamercovered{transparent}
}

\usepackage[english]{babel}
\usepackage[latin1]{inputenc}
\usepackage[scaled=.90]{helvet}
\usepackage{courier}
\usepackage[T1]{fontenc}
\usepackage{comment}
%usepackage{appendixnumberbeamer}
\usepackage{amsmath}
\usepackage{bm}
\usepackage{pgfpages}
\usepackage{multicol}
% citations
\usepackage{natbib}
\bibpunct{(}{)}{;}{a}{,}{,}
\def\citeapos#1{\citeauthor{#1}'s (\citeyear{#1})}
\renewcommand{\bibsection}{\subsubsection*{\bibname } }

\input{latex-math/basic-math.tex}
\input{latex-math/basic-ml.tex}
\input{latex-math/ml-infotheory.tex}
\newcommand{\etpx}{H(X)}
\newcommand{\etpy}{H(Y)}
\newcommand{\etpxy}{H(X, Y)}
\newcommand{\etpycx}{H(Y|X)}
\newcommand{\etpxcy}{H(X|Y)}
\newcommand{\mixy}{I(X; Y)}
\newcommand{\miyz}{I(Y;Z)}
\newcommand{\miyzcx}{I(Y;Z|X)}
\newcommand{\mixyz}{I(X;Y;Z)}
\newcommand{\mixycz}{I(X;Y|Z)}
\newcommand{\px}{p(x)}
\newcommand{\py}{p(y)}
\newcommand{\pz}{p(z)}
\newcommand{\pxy}{p(x,y)}
\newcommand{\pxz}{p(x, z)}
\newcommand{\pyz}{p(y, z)}
\newcommand{\pxyz}{p(x, y, z)}
\newcommand{\pxycz}{p(x, y | z)}
\newcommand{\pxcz}{p(x|z)}
\newcommand{\pycz}{p(y|z)}

\title[]{\textbf{Exercise of Supervised Learning:\\Information Theory Part 2}}

%\subtitle{}

% - Use the \inst{?} command only if the authors have different
%   affiliation.
%\author{F.~Author\inst{1} \and S.~Another\inst{2}}
\author{Yawei Li} 

% - Use the \inst command only if there are several affiliations.
% - Keep it simple, no one is interested in your street address.
\institute[LMU]
{
\\
  \texttt{yawei.li@stat.uni-muenchen.de}
}

\date{February 2, 2024}



\begin{document}



\begin{frame}[noframenumbering, plain]
\titlepage

\end{frame}

\begin{frame}{Exercise 1: Entropy}
	A fair die is rolled at the same time as a fair coin is tossed. Let $A$ be the number on the upper surface of the dice and let $B$ describe the outcome of the coin toss, where 
	\begin{align*}
		B = \begin{cases}
			1, \quad & \text{head,} \\
			0 \quad & \text{tail.}
		\end{cases}
	\end{align*}
	Two random variables $X$ and $Y$ are given by $X = A + B$ and $Y = A - B$, respectively.
	\vspace{10pt}
	
	(a) Calculate the entropies $H(X)$ and $H(Y)$, the conditional entropies $H(Y|X)$ and $H(X|Y)$, the joint entropy $H(X, Y)$ and the mutual information $I(X;Y)$.
\end{frame}

\begin{frame}{Exercise 1 (a)}
(a) Calculate the entropies $H(X)$ and $H(Y)$, the conditional entropies $H(Y|X)$ and $H(X|Y)$, the joint entropy $H(X, Y)$ and the mutual information $I(X;Y)$.

\textbf{Solution}: Let $a, b, x,$ and $y$ denote the realizations of $A, B, X$ and $Y$, respectively.

Note that each event $(a,b)$ is associated with \textbf{exactly one} event $(x,y)$. Why?

Because given a pair $(x, y)$, $a = 0.5 (x + y)$ and $b = 0.5 (x - y)$ are unique. 
So the joint probability
\begin{align*}
	p_{AB}(a, b) = p_{XY}(x, y) = \frac{1}{6} \cdot \frac{1}{2} = \frac{1}{12}
\end{align*}

So, the joint entropy:
\begin{align*}
	H(X, Y) &= - \sum_{x, y} p_{X, Y}(x,y) \log_2 p_{X, Y}(x, y) \\ 
	&= - 12 \cdot \frac{1}{12} \log_2 \frac{1}{12} \\ 
	&= 2 + \log_2 3
\end{align*}
	
\end{frame}

\begin{frame}{Exercise 1 (a): Continued}
	\begin{columns}
		\begin{column}{0.5\textwidth}
			\begin{table}
				\begin{tabular}{c l c}
					\hline
					$x$ & Events $(a,b)$ & $p_X(x)$ \\
					\hline
					1 & $(1, 0)$	 & $1/12$ \\
					2 & $(2, 0), (1, 1)$	 & $1/6$ \\
					3 & $(3, 0), (2, 1)$	 & $1/6$ \\
					4 & $(4, 0), (3, 1)$	 & $1/6$ \\
					5 & $(5, 0), (4, 1)$	 & $1/6$ \\
					6 & $(6, 0), (5, 1)$	 & $1/6$ \\
					7 & $(6, 1)$	 & $1/12$ \\
					\hline
				\end{tabular}
			\end{table}
		\end{column}
		
		\begin{column}{0.5\textwidth}
			\begin{table}
				\begin{tabular}{c l c}
					\hline
					$y$ & Events $(a,b)$ & $p_Y(y)$ \\
					\hline
					0 & $(1, 1)$	 & $1/12$ \\
					1 & $(1, 0), (2, 1)$	 & $1/6$ \\
					2 & $(2, 0), (3, 1)$	 & $1/6$ \\
					3 & $(3, 0), (4, 1)$	 & $1/6$ \\
					4 & $(4, 0), (5, 1)$	 & $1/6$ \\
					5 & $(4, 0), (6, 1)$	 & $1/6$ \\
					6 & $(6, 0)$	 & $1/12$ \\
					\hline
				\end{tabular}
			\end{table}
		\end{column}
	\end{columns}
	
	\begin{align*}
		\etpx &= \sum_x p_X(x) \log_2 p_X(x) = \frac{7}{6} + \log_2 3. \\
		\etpy &= \sum_x p_Y(y) \log_2 p_Y(y) = \frac{7}{6} + \log_2 3. 
	\end{align*}

\end{frame}

\begin{frame}{Exercise 1 (a): Continued}
	\textbf{Disclaimer: the basic formulas like the following ones need to be either remembered or written on your cheatsheet. They will not be given on the exam sheets.}
	
	The conditional entropies
	\begin{align*}
		\etpxcy &= \etpxy - \etpy = \frac{5}{6} \\
		\etpycx & = \etpxy - \etpx = \frac{5}{6} \\
	\end{align*}
	The mutual information 
	\begin{align*}
		\mixy = \etpx - \etpxcy = \frac{1}{3} + \log_2 3.
	\end{align*}
\end{frame}

\begin{frame}{Exercise 1 (b)}
	(b) Show that, for independent discrete random variables $X$ and $Y$, 
	\begin{align*}
		I(X; X + Y) - I(Y; X + Y) = H(X) - H(Y).
	\end{align*}
	Solution: 
	\begin{align*}
		I(X; X + Y) - I(Y; X + Y) &= H(X+Y) - H(X+Y |X) - H(X+Y) + H(X+Y|Y) \\
		&= H(X+Y|Y) - H(X+Y|X)
	\end{align*}
	Note that $p(x + y | x) = p(y|x)$, because $x$ is given, so if we observe $x+y$, we can immediately infer the value of $y$. In this case where $X$ is observed, there is an one-to-one mapping between $X+Y$ and $Y$.
	
	So, $H(X+Y|X) = H(X|Y)$, and $H(X+Y|Y) = H(X|Y)$. Hence,
	\begin{align*}
		I(X; X + Y) - I(Y; X + Y) &= H(X|Y) - H(Y|X) \\
			& = H(X) - H(Y) \\
	\end{align*}
	
\end{frame}

\begin{frame}{Exercise 2 (a)}
	Let $X, Y$ and $Z$ be three discrete random variables. The mutual information of $X$, $Y$ and $Z$ is defined as:
	\begin{align*}
		\mixyz = \sum_x \sum_y \sum_z \pxyz \log \left(\frac{\pxy \pxz \pyz}{\px \py \pz \pxyz} \right)
	\end{align*}
	(a) Prove the lemma: $\mixyz = \mixy - \mixycz$. Note that the conditional mutual information is defined as:
	\begin{align*}
		\mixycz = \sum_x \sum_y \sum_z \pz \pxycz \log \frac{\pxycz}{\pxcz \pycz}
	\end{align*}
	Hint: Starting from the right hand side of the equation in the lemma.
\end{frame}

\begin{frame}{Exercise 2 (a): Continued}
	\small
	\begin{align*}
		\mixy - \mixycz 
		&= \sum_x \sum_y \pxy \log \frac{\pxy}{\px \py} - \sum_z \sum_x \sum_y \pz \pxycz \log \frac{\pxycz}{\pxcz \pycz} \\
		&= \sum_x\sum_y\sum_z \pxyz \log \frac{\pxy}{\px \py} \\ 
		&\qquad - \sum_x \sum_y \sum_z \pxyz \log \frac{\pxycz \pz^2}{\pxcz \pycz \pz^2} \\
		&= \sum_x\sum_y\sum_z \pxyz \log \frac{\pxy}{\px \py} - \sum_x \sum_y \sum_z \pxyz \log \frac{\pxyz \pz}{\pxz \pyz} \\
		&= \sum_x \sum_y \sum_z \pxyz \log \left( \frac{\pxy \pxz \pyz}{\px \py \pz \pxyz} \right) \\
		&= \mixyz
	\end{align*}
	
\end{frame}

\begin{frame}{Exercise 2 (b)}
	(b) Prove the following relation with the above lemma:
	\begin{align*}
		\mixy = \mixycz + \miyz - \miyzcx.
	\end{align*}
	Solution: Using the lemma on the conditional MI, we obtain:
	\begin{align*}
		& \mixycz + \miyz - \miyzcx \\
		&= \mixy - \mixyz + \miyz - \miyz + \mixyz \\
		&= \mixy.
	\end{align*}
\end{frame}

\begin{frame}{Exercise 3}
	\textbf{Show the standard solution.}
\end{frame}

\end{document}
