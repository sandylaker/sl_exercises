%% LaTeX Beamer presentation template (requires beamer package)
%% see http://bitbucket.org/rivanvx/beamer/wiki/Home
%% idea contributed by H. Turgut Uyar
%% template based on a template by Till Tantau
%% this template is still evolving - it might differ in future releases!

%% Template edited by Panagiotis Adamopoulos {padamopo}@stern.nyu.edu

\documentclass[aspectratio=169]{beamer}
 
\mode<presentation>
{
\usetheme{NYU}

\setbeamercovered{transparent}
}

\usepackage[english]{babel}
\usepackage[latin1]{inputenc}
\usepackage[scaled=.90]{helvet}
\usepackage{courier}
\usepackage[T1]{fontenc}
\usepackage{comment}
%usepackage{appendixnumberbeamer}
\usepackage{amsmath}
\usepackage{bm}
\usepackage{pgfpages}
% citations
\usepackage{natbib}
\bibpunct{(}{)}{;}{a}{,}{,}
\def\citeapos#1{\citeauthor{#1}'s (\citeyear{#1})}
\renewcommand{\bibsection}{\subsubsection*{\bibname } }

\input{latex-math/basic-math.tex}
\input{latex-math/basic-ml.tex}
\input{latex-math/ml-infotheory.tex}

\newcommand{\kld}[1]{D_{KL}\left(#1\right)}
\newcommand{\Ef}{\mathbb{E}_f}
\newcommand{\Varf}{\mathrm{Var}_f}

\title[]{\textbf{Supervised Learning: Exercise of \\ Information Theory Part 1}}

%\subtitle{}

% - Use the \inst{?} command only if the authors have different
%   affiliation.
%\author{F.~Author\inst{1} \and S.~Another\inst{2}}
\author{Yawei Li} 

% - Use the \inst command only if there are several affiliations.
% - Keep it simple, no one is interested in your street address.
\institute[LMU]
{
\\
  \texttt{yawei.li@stat.uni-muenchen.de}
}

\date{Date}


% This is only inserted into the PDF information catalog. Can be left
% out.
\subject{Subject}



% If you have a file called "university-logo-filename.xxx", where xxx
% is a graphic format that can be processed by latex or pdflatex,
% resp., then you can add a logo as follows:

% \pgfdeclareimage[height=0.5cm]{university-logo}{university-logo-filename}
% \logo{\pgfuseimage{university-logo}}



% Delete this, if you do not want the table of contents to pop up at
% the beginning of each subsection:
%\AtBeginSubsection[]
%{
%\begin{frame}<beamer>
%\frametitle{Outline}
%\tableofcontents[currentsection,currentsubsection]
%\end{frame}
%}

% If you wish to uncover everything in a step-wise fashion, uncomment
% the following command:

%\beamerdefaultoverlayspecification{<+->}

\begin{document}



\begin{frame}[noframenumbering, plain]
\titlepage

\end{frame}


\begin{frame}{Exercise 1: Kullback-Leibler Divergence}
	(a) You want to approximate the binomial distribution with $n$ numbers of trials and probability $p$ with a Gaussian distribution with mean $\mu$ and variance $\sigma^2$. To find a suitable distribution you investigate the Kullback-Leibler divergence (KLD) in terms of the parameters $\thetab = (\mu, \sigma^2)^T$.
	\begin{enumerate}
		\item Write down the KLD for the given setup.
		\item Derive the gradients w.r.t. $\thetab$.
		\item Is there an analytic solution for the optimal parameter setting? If yes, derive the corresponding solution. If no, give a short reasoning.
		\item Independent of the previous exercise, state a numerical procedure to minimize the KLD.
	\end{enumerate}
\end{frame}

\begin{frame}{Solution to Question (a.1)}
	\small
	Question: Write down the KLD for the given setup.
	
	\begin{itemize}
		\item Let $f$ be pmf of $Bin(n, p)$, we know that $\Ef[X] = np$ and $\Varf[X] = np (1 - p)$.
		\item Let $q$ be the density of $\mathcal{N}(\mu, \sigma^2)$.
		\item The KL divergence between $f$ andd $q$ is
		\begin{align*}
			\kld{f \ || \ q} 
			&= \Ef \left[ \log \frac{f(X)}{q(X; \thetab)} \right] = \underbrace{\Ef \left[ \log f(X) \right]}_{c_1} - \Ef [\log q(X|\thetab)] \\
			&= c_1 - \Ef \left[ \log \left( \frac{1}{\sqrt{2 \pi \sigma^2}} \exp \left( - \frac{1}{2\sigma^2} (X - \mu)^2 \right)  \right) \right] \\
			&= c_1 - \Ef \left[ - \underbrace{\frac{1}{2} \log(2\pi)}_{c_2} - \frac{1}{2} \log (\sigma^2) - \frac{1}{2\sigma^2} (X - \mu)^2 \right]
		\end{align*}
	\end{itemize}
	
\end{frame}


\begin{frame}{Solution to Question (a.1): Continued}
	\small
	\begin{align*}
		\kld{f \ || \ q} &= \underbrace{c_1 + c_2}_{c_3}  + \frac{1}{2}\log(\sigma^2) + \frac{1}{2\sigma^2} \Ef[(X - \mu)^2] \\
		&= c_3 + \log(\sigma) + \frac{1}{2\sigma^2} \left( \underbrace{\Ef[X^2]}_{\Varf[X] + \Ef[X]^2} - 2\mu \Ef[X] + \mu^2 \right) \\
		&= c_3 + \log(\sigma) + \frac{1}{2\sigma^2} (np(1 - p) + (np)^2 - 2\mu np + \mu^2) \\
		&= c_3 + \log(\sigma) + \frac{1}{2\sigma^2}(np(1-p) + (np -\mu)^2)
	\end{align*}
\end{frame}

\begin{frame}{Solution to Question (a.2)}
	\small
	Question: Derive the gradients w.r.t. $\thetab$.
	\begin{align*}
		\frac{\partial \kld{ f \ || \ q}}{\partial \mu} &= \frac{\partial}{\partial \mu} \frac{(np - \mu)^2}{2 \sigma^2} = \frac{2 (np - \mu ) \cdot ( -1)}{2\sigma^2} = \frac{ \mu - np}{\sigma^2}
	\end{align*}
	\begin{align*}
		\frac{\partial \kld{f \ || \ q}}{\partial \sigma^2} 
		&= \frac{\partial}{\partial \sigma^2} 
		\left( \frac{1}{2} \log (\sigma^2) + \frac{1}{2 \sigma^2} (np(1 - p) + (np - \mu)^2)  \right) \\
		&= \frac{1}{2\sigma^2} - \frac{1}{2 (\sigma^2)^2}(np (1- p) + (np -\mu)^2) \\
		&= \frac{1}{2\sigma^2} - \frac{1}{2 \sigma^4} \Ef[(X -\mu)^2]
	\end{align*}
\end{frame}

\begin{frame}{Solution to Question (a.3)}
	\small
	Question: Is there an analytic solution for the optimal parameter setting? If yes, derive the corresponding solution. If no, give a short reasoning.
	
	Solution: Yes, there is. By setting the partial derivatives to zero we can identify the critical points.
	\begin{align*}
		\frac{\partial \kld{f \ || \ q}}{\partial \mu} = 0 &\Leftrightarrow \hat{\mu} = np = \Ef[X] 
	\end{align*}
	\begin{align*}
		\frac{\partial \kld{f \ || \ q}}{\partial \sigma^2} = 0 & \Leftrightarrow \frac{1}{2\hat{\sigma}^2} = \frac{1}{2 \hat{\sigma}^2} \Ef[(X - \hat{\mu})^2] \\
		& \Leftrightarrow \hat{\sigma}^2 = \Ef[(X - \Ef[X])^2] = \Varf[X] = np(1-p)
	\end{align*}
	In order to prove that critical point is a minimum, we need to check the Hessian matrix w.r.t. $(\mu, \sigma^2)$ is positive definite at $(\mu, \hat{\sigma}^2)$.
\end{frame}

\begin{frame}{Solution to Question (a.3): Continued}
	The Hessian matrix is:
	\begin{align*}
		\nabla^2 = \begin{pmatrix}
			\frac{1}{\sigma^2} & (np -\mu) \frac{1}{\sigma^4} \\
			(np -\mu) \frac{1}{\sigma^4} & - \frac{1}{2\sigma^4} + \frac{1}{\sigma^6} \Ef [(X -\mu)^2]
		\end{pmatrix}
	\end{align*}
	Evaluated at $(\hat{\mu}, \hat{\sigma}^2)$, we obtain
	\begin{align*}
		\nabla^2 \bigg|_{(\hat{\mu}, \hat{\sigma}^2)} 
		&= \begin{pmatrix}
			\frac{1}{np(1-p)} & 0 \\
			0 & \frac{1}{2 (np(1-p))^2} \\
			\end{pmatrix} \\
		&= \begin{pmatrix}
			\frac{1}{\Varf(X)} & 0 \\
			0 & \frac{1}{2 \Varf(X)^2} \\
		\end{pmatrix}
	\end{align*}
	So the Hessian at $(\hat{\mu}, \hat{\sigma}^2)$ is P.S.D.
\end{frame}

\begin{frame}{Question (b), (c) and (d)}
	(b) Sample points according to the true distribution and visualize the KLD for different parameter settings of the Gaussian distribution (including the optimal one if available).
	
	\vspace{20pt}
	(c) Create a surface plot with axes $n$ and $p$ and colour value equal to the KLD for the optimal normal distribution.
	
	\vspace{20pt}
	(d) Based on the previous result, how can be behavior for varing $p$ and $n$ be explained, respectively?
	
	\vspace{20pt}
	\textbf{Solution: Show the code.}
\end{frame}


\begin{frame}{Exercise 2: The Convexity of KL Divergence}
	Let $p$ and $q$ be the PDFs of a pair of absolutely continuous distributions. 
	
	(a) Prove that the KL divergence is convex in the pair $(p, q)$, i.e., 
	$$\kld{\lambda p_1 (1 - \lambda) p_2 \ || \ \lambda q_1 + (1-\lambda) q_2} \leq \lambda \kld{p1 \ || \ q_1} + (1 - \lambda) \kld{p_2 \ || \ q_2},$$
	where $(p_1, q_1)$ and $(p_2, q_2)$ are two pairs of distribution and $0 \leq \lambda \leq 1$.
	
	\vspace{20pt}
	Hint: you can use the log sum inequality, namely that $(a_1 + a_2) \log \left( \frac{a_1 + a_2}{b_1 + b_2} \right) \leq a_1 \log \frac{a_1}{b_1} + a_2 \log \frac{a_2}{b_2}$ holds for $a_1, a_2, b_1, b_2 \geq 0$.
\end{frame}

\begin{frame}{Solution to Question (a)}
	\small
	\begin{align*}
		& \kld{\lambda p_1 (1 - \lambda) p_2 \ || \ \lambda q_1 + (1-\lambda) q_2}  \\
		&= \int_\Xspace \left( (\lambda p_1(x) + (1 - \lambda) p_2(x)) \log \frac{\lambda p_1(x) + (1 - \lambda) p_2(x)}{\lambda q_1(x) + (1 - \lambda) q_2(x)} \right) \mathrm{d}x \\
		&\leq \int_\Xspace \left( \lambda p_1(x) \log \frac{p_1(x)}{q_1(x)} + (1-\lambda) p_2(x) \log \frac{(1 - \lambda) p_2(x)}{(1-\lambda) q_2(x)} \right) \mathrm{d}x \\
		&= \lambda \int_\Xspace \left(p_1(x) \log \frac{p_1(x)}{q_1(x)} \right) \mathrm{d}x + (1 - \lambda) \int_\Xspace \left( p_2(x) \log \frac{p_2(x)}{q_2(x)}\right) \mathrm{d}x \\
%		&= \lambda \kld{p_1 \ || \ q_1} + (1 - \lambda) \kld{p_2 \ || \ q_2}.
	\end{align*}
\end{frame}


\end{document}
